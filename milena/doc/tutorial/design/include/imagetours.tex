%% Outline:
%% This file gives a quick overview of the following image types:

%% image2d<T>
%% mmap_image2d<T>
%% file_image2d<T>
%% tiled_image
%% fun_image<S,F>
%% flat_image<S,T>
%% pkey_image<P,T>
%% pvkeys_image<P,T>

%% graph-based images...

%% rle_image<P,T>
%% sparse_image<P, T>
%% value_enc_image<P, T>

%% f_image<F,I>
%% f_images<F,n,I>
%% lut_image<I,T>

%% sub_image
%% image_if
%% hexa<I>
%% translated_image


\section{Images tour}

\subsection{Introduction}


In the Milena library, an image can be seen as an application
$site$ to $value$.
A $site$ is a localized object in space.
Points $1D$, $2D$ or $3D$ are the sites objects commonly used in the
library.
%%However, the $site$ concept allows Milena to deal with complicated image type
%%(for instance see the \verb+graph_image+ type).

An image is composed by a set of localized objects ($sites$), the definition domain of the image.
A value is associated to each site of the image. This is the destination domain
of the image.
To access to a value  localized at the site p in an image named $ima$, we
just use the mathematics notation: \verb+ima(p)+.

Obviously, every image types of Milena provide an access $site$ to $value$,
through \verb+ima(p)+. Yet, depending on the image type, the values can be
stored in different way in memory.
For instance, in the \verb+image2d+, all the values are stored directly in a
memory buffer.
On the contrary, some images compress the space taken by the destination set
like the \verb+rle_image+.
These images use runs.
A run is a succession of points that share the same value. It is encoded by
a point (the beginning of the run), and an integer (the length of the run).
Runs allows \verb+rle_image+ to gain space in memory.\\

%% Explain what is a run

%% Explain the different template parameters T/S/F...
All the image types are parametrized by different static parameters.
In this document, we will used the following naming convention for the
image types parameters:
\begin{itemize}
\item{\verb+T+:} represents an image $value$ type.
\item{\verb+S+:} represents a type of a $sites$ $set$ type.
\item{\verb+F+:} is a type of a function $site$ to $value$.
\item{\verb+P+:} represents a $site$ type.
\item{\verb+I+:} represents an $image$ type.
\end{itemize}

\subsection{Primary images}

%% Primary image definition

Primary images are a major category of the Milena image types.
Primary images are not based on another image type.
They are sufficient to define themselves.
Thus, a primary image types directly holds its data (values) in memory.


\begin{itemize}

\item{\verb+image2d<T>+}: rectangular images based on a 2d grid with all its
values stored in a buffer in memory. An extended domain is added to the image
domain.

Related image type: \verb+image1d<T>+, \verb+image3d<T>+.


\item{\verb+mmap_image2d<T>+:}  this image2d-like type has its values data
``memory mapped''.

Related image types: \verb+mmap_image1d<T>+, \verb+mmap_image3d<T>+.

\item{\verb+file_image2d<T>+:} this image2d-like type has its values data
stored in a file.

Related image types: \verb+file_image1d<T>+, \verb+file_image3d<T>+.

\item{\verb+tiled_image<T>+:} FIXME.

\item{\verb+fun_image<S, F>+:} image type defined by a domain \verb+S+ and a
function f of type \verb+F+: $site \rightarrow value$.
f associates some values to each site composing the image.

\item{\verb+flat_image<S, T>+:} defined by a domain and a value $v$.
All the sites composing a flat image share the same value $v$.

\item{\verb+pkey_image<P, T>+:} defined by couples $(p,v)$ where the $site$ p
is a key to retrieve the values v. A map, a sorted associative container, is
used to store the couples $(p,v)$.

\item{\verb+pvkeys_image<P, T>+:} defined by couples $(p,v)$ where  the site p is a key, and at the same time, coherently maintaining couples
($v$, list of $p$) where the value v is the key.


%% FIXME Graph Image
\item{\verb+graph_image+:} FIXME

%% FIXME what is a run...
\item{\verb+rle_image<P, T>+:} defined by couples $(r, v)$, where $r$ is a run
and $v$ a value. A value is associated to each run.

\item{\verb+sparse_image<P, T>+:} defined by couples ($r$, list of $v$) where a
run $r$ is associated to a list of values. A value is associated to each
point composing a run.

\item{\verb+value_enc_image<P, T>+:} defined by couples ($v$, list of $r$).
A list of run is associated to every values composing the image. This is a
value-driven image.

\end{itemize}

\subsection{Morphers}

%% Morpher definition
A morpher transforms a type into another one.
It can be seen as an extension of the decorator design pattern.
Morphers are a non-intrusive way to add or modify some behaviors in an
existing class.
Here, the image morphers rest upon another image (the input image(s) which
will be transformed).
Since, it extends an image, an image morpher is also an image type.

The Milena library provides different kinds of morphers.
Domain morphers only modify the domain (the set of points/sites composing the image) of the input image.
Value morphers only change the input image values (cast the values into another
type\ldots{})
Identity morphers don't modify either the definition domain or the image
values.



\subsubsection{Domain morphers}

\begin{itemize}

\item{\verb+sub_image<I, S>+:} restrict an image to a given sites set.

\item{\verb+image_if<I, F>+:} restrict the input image domain to all the image
sites that satisfy a condition expressed by a function.

\item{\verb+hexa<I>+:}  make a hexagonal mesh of the input image (the input
image must be an image in two dimension).

\item{\verb+translated image<I, T>+:} apply a translation of $dp$, a given
delta-point to all the sites composing the input image.
\end{itemize}

\subsubsection{Value morphers}

\begin{itemize}

\item{\verb+f_image<F, I>+:} transform the image values through a "function".

\item{\verb+f_images<F, n, I>+:} takes $n$ values from $n$ input images,
and transform them through a function (that return only one value).

\item{\verb+lut_image<I, T>+:} use a translation table, to do a binding between the input image values and the morpher output values.

\end{itemize}


\subsubsection{Identity morphers}

\begin{itemize}
\item FIXME...
\end{itemize}


\subsection{Data access:}

\begin{tabular}{ccl}

image type & ima(p) & Details\\
\hline

\verb+image2d<T>+ &
\lstinline+values[p.row][p.col]+ &
\verb+values+ is a buffer containing all the image\\
&
&
values.\\

\\

\verb+fun_image<S, F>+ &
\lstinline+f(p)+ &
\verb+f+ is the function associated to the image.\\

\verb+flat_image<T>+ &
\lstinline+v+ &
\verb+v+ is the flat value.\\

\\

\verb+pkey_image<P, T>+ &
\lstinline+map[p]+ &
\verb+map+ is a table that associates a value to\\
&
&
each key \verb+p+.\\

\verb+pkeys_image<P, T>+ &
\lstinline+map[p]+ &
same as \verb+pkey_image<P, T>+.\\

\\

\verb+rle_image<P, T>+ &
\lstinline+values[r.in_index()]+ &
\verb+values+ is an array that associates a value\\
&
&
to each run.\\
&
&
\verb+in_index+ returns the index of the run in\\
&
&
the image.\\

\verb+sparse_image<P, T>+ &
\lstinline+values[r.in_index()][r.out_index()]+ &
\verb+values+ is an 2D array that associates a\\
&
&
value to each point in a run.\\
&
&
\verb+in_index+ returns the index of the run in\\
&
&
the image.\\
&
&
\verb+out_index+ returns the index of the current\\
&
&
point in the run.\\

\verb+value_enc_image<P, T>+ &
\lstinline+values[r.in_index()]+ &
\verb+values+ is an array that associates a values\\
&
&
to each run.\\

\\

\verb+f_image<F, I>+ &
\lstinline+f(ref(p))+ &
\verb+f+ is the function associated to the morpher.\\
&
&
\verb+ref+ is the underlying image.\\

\verb+f_images<F, n, I>+ &
\lstinline+f(ref_1(p), ref_2(p), ... ref_n(p))+ &
\verb+f+ is the function associated to the morpher.\\
&
&
\verb+ref_n+ are the underlying images.\\


\verb+lut_image<I, T>+ &
\lstinline+lut[ref(p)]+ &
\verb+lut+ is the morpher translation table.\\
&
&
\verb+ref_n+ are the underlying images.\\

\end{tabular}


