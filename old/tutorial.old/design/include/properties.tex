\section{Properties}



\subsection{Notation}

A property has several values.
Furthermore the values of a property can be seen as a hierarchy.
Hence, we need to define a notation to represent these hierarchies.
\begin{itemize}

\item \verb+category: primary+ means that \verb+primary+ is a value of the
property category.

\item
\begin{verbatim}
category: any
           |
           +--- primary
           |
           +--- morpher
\end{verbatim}
\end{itemize} It means that \verb+primary+ and \verb+morpher+ inherit from
\verb+any+

\subsection{Properties defining the image}

%%%%%%%%%%%%%%%%%%
% Category       %
%%%%%%%%%%%%%%%%%%
\subsubsection{category}
\index{Property: category}


\begin{verbatim}
  category: any
             |
             +--- primary
             |
             +--- morpher
                     |
                     +--- domain_morpher
                     |
                     +--- value_morpher
                     |
                     +--- identity_morpher
\end{verbatim}


This property specifies the category of an image type.
Primary images are instantiated in the first place; they don't need any
prior image type definition.
Morpher images transform an image type into another one.
Morphers are a non-intrusive way to add or modify some behaviors in an
existing class.


%% FIXME image morpher

%%%%%%%%%%%%%%%%%%
% Size           %
%%%%%%%%%%%%%%%%%%
\subsubsection{size}
\index{Property: size}

\begin{verbatim}
  size: any
         |
         +--- regular
         |
         +--- huge
\end{verbatim}


The size property gives an indication about the memory needed by an image.
A $huge$ image is an image which cannot be fully stored in the RAM.
Accessing a $huge$ image is slower than accessing a $regular$ image.
Indeed, to access to a $huge$ image, we first need to load in the RAM a part of
the image from another storage device.

\subsection{Properties related to the image localization}

%%%%%%%%%%%%%%%%%%
% Localization   %
%%%%%%%%%%%%%%%%%%
\subsubsection{localization}
\index{Property: localization}

\begin{verbatim}
 localization: /any/
                 |
                 + -- none
                 |
                 + -- space
                       |
                       + -- /grid/
                              |
                              + -- isotropic_grid
                              |     |
                              |     + -- basic_grid
                              |
                              + -- anisotropic_grid

\end{verbatim}

This property defines the underlying support of the images.
The support of an image describes the image geometry and the relationship
between its sites.
An image on a non-localized space has this property set to $none$.
An image on a localized space has this property set to $space$.
An image based on a isotropic grid has this property set to $isotropic\_grid$
An image based on a anisotropic grid has this property set to
$anisotropic\_grid$
An image based on a aligned and orthogonal grid has its property set to
$basic\_grid$

If an image has its \verb+localization+ set to $space$, the user must define
the neighborhood relationships between the sites.
Otherwise, the image underlying grid defines the neighborhood relationships.


%% FIXME Image de different type de grid

%%%%%%%%%%%%%%%%%%
% Space          %
%%%%%%%%%%%%%%%%%%
\subsubsection{space}
\index{Property: space}

\begin{verbatim}
  dimension:  any
               |
               +--- none
               |
               +--- some
                      |
                      +--- one_d
                      |
                      +--- two_d
                      |
                      +--- three_d
\end{verbatim}

This property specifies the dimension of the image.
An image in Milena can either be in one dimension, two dimensions or three dimensions.
An image can also have no dimension specified.



\subsection{Property related to the image values}


%%%
%%% Data storage
%%%
\subsubsection{value\_access}
\index{Property: value\_access}

\begin{verbatim}
value_access: any
               |
               +--- direct
               |
               +--- indirect
                       |
                       +--- computed
\end{verbatim}


Image values can either be computed on the fly by a function or stored in
memory. If an image type has a direct access on the values in memory, it is
possible to take a reference of them. In this case, this property is set to
$direct$. In the other case, this property is set $indirect$. When an image
computes its values on the fly, this property is refined to $computed$.


\subsubsection{value\_storage}
\index{Property: value\_storage}

\begin{verbatim}
value_storage: any
                |
                +--- /memory/
                |      |
                |      +--- singleton
                |      |
                |      +--- one_block
                |      |
                |      +--- piecewise
                +--- irrelevant
\end{verbatim}

This property indicates how the values are stored in memory, when it is the
case. If the values aren't stored in the $memory$, this property is set to
$irrelevant$. If this property is set to $singleton$, only one value is store
in memory for the whole image (this the case of the \verb+flat_image+ type).
If this value is set to $one\_block$, all the image values are stored in a
linear buffer in memory. And, an image is $piecewise$, when it's value are
stored in several buffers in memory.


\subsubsection{value\_alignment}
\index{Property: value\_alignment}

\begin{verbatim}
value_alignment: any
                   |
                   +--- with_grid
                   |
                   +--- not_aligned
                   |
                   +--- irrelevant
\end{verbatim}

This property indicates if the image values are ordered the same way in memory
than when we run over them through a \verb+fwd_piter+.
When this is the case, the image values are aligned with the image underlying
grid, this property is set to $with\_grid$.
If the values are not ordered the same way in memory than the \verb+fwd_piter+
order, this property is set to $not\_aligned$.
If the image values are not stored in memory, this property is set to
$irrelevant$.


%%%%%%%%%%%%%%%%%%%%%%%%
% access               %
%%%%%%%%%%%%%%%%%%%%%%%%

\subsubsection{pw\_io}
\index{Property: pw\_io}

\begin{verbatim}

  pw_io:  any
          |
          +--- read
          |
          +--- read_write
\end{verbatim}


All the image types in Milena provide an access to its values through
\verb+ima(p)+ where \verb+ima+ is an image and \verb+p+ a site.

This property describes the read/write accessibility of this access.
If the image is read only, \verb+pw_io+ is set to $read$.
If the image is readable and writable, \verb+pw_io+ is set to $read\_write$.\\


%%%%%%%%%%%%%%%%%%%%%%%%
%   Speed              %
%%%%%%%%%%%%%%%%%%%%%%%%

%% FIXME detais fast image..., add normal image
\subsubsection{speed}
\index{Property: speed}

\begin{verbatim}

  speed:  any
           |
           +--- slow
           |
           +--- fast
           |
           +--- fastest
\end{verbatim}

  This property gives some information about the time needed to access a
value from a site through \verb+ima(p)+ where \verb+ima+ is an image and
\verb+p+ a site.

If the \verb+speed+ property is equal to $slow$, $ima(p)$ complexity is greater
than $O(1)$.
If this property is equal to $fast$, $ima(p)$ complexity is in
$O(1)$
If this property is equal to $fastest$, $ima(p)$ complexity is
in $O(1)$. Furthermore, in this case, $ima$ has an extended domain, and $ima$
values have a pointer semantics (it is possible to directly iterate on the
image pixels).

\subsubsection{vw\_io}
\index{Property: vw\_io}

%% FIXME functions that modify globaly all the values...
\begin{verbatim}
  vw_io:  any
           |
           +--- /some/
           |        |
           |        +--- vw_read
           |        |
           |        +--- vw_read_write
           |        |
           |        +--- vw_global_read_write
           +--- none
\end{verbatim}

This property defines the value wise access of an image.
If \verb+vw_io+ is set to $vw\_none$, the associated image type does not
 provide any value wise access.
If an image type provides a readable (or writable)
value access, this property is set to $vw\_read$ (or $vw\_read\_write$).
If the image provide a value wise access that allows the user to change
globally all the image values through a function $value$ to $value$,
this property is set to $vw\_global\_read\_write$.

%% FIXME define the interface associated

The method provided for this value access depend on the \verb+vm_set+ property.


\subsubsection{vw\_set}
\index{Property: vw\_set}

\begin{verbatim}

  vw_set:  any
           |
           +--- /some/
           |      |
           |      +--- uni
           |      |
           |      +--- multi
           +--- none
\end{verbatim}


This property precises what kind of value access is provided by an image type.
If an image type does not provide any value access (\verb+vw_io+ set to $none$)
, \verb+vw_set+ is set to $none$.
Otherwise, \verb+vw_set+ is set to $uni$ if the image provides an
access to the mathematical set of values taken by the image (through the method
\verb+value_taken+).
If \verb+vw_set+ is set to $multi$, the image provides an access to the values
taken in memory (cells) (through the method \verb+cells+). Note that a value
can be present twice in the cells set.



\subsection{Properties related to the extended domain}

%%%%%%%%%%%%%%%%%%%%%%%%
%   ext_domain         %
%%%%%%%%%%%%%%%%%%%%%%%%
\subsubsection{ext\_domain}
\index{Property: ext\_domain}

\begin{verbatim}
  ext_domain: any
               |
               +--- none
               |
               +--- some
                     |
                     +--- fixed
                     |
                     +--- infinite
                     |
                     +--- extendable
\end{verbatim}

This property indicates if the image has an extended domain.
An extended domain grows the image domain by adding dummy values
on the image border.
\begin{table}[!h]
  \begin{center}
    \begin{tabular}{c|c|c|c|c|c}
      {\textbf{\textcolor{green}{1}}} & {\textbf{\textcolor{green}{1}}} &
      {\textbf{\textcolor{green}{1}}} & {\textbf{\textcolor{green}{1}}} &
      {\textbf{\textcolor{green}{1}}} & {\textbf{\textcolor{green}{1}}}\\
      \hline
      {\textbf{\textcolor{green}{1}}} & {\textbf{\textcolor{green}{1}}} &
      {\textbf{\textcolor{green}{1}}} & {\textbf{\textcolor{green}{1}}} &
      {\textbf{\textcolor{green}{1}}} & {\textbf{\textcolor{green}{1}}}\\
      \hline
      {\textbf{\textcolor{green}{1}}} &
      {\textbf{\textcolor{green}{1}}} & {\textbf{\textcolor{red}{12}}} &
      {\textbf{\textcolor{red}{2}}} & {\textbf{\textcolor{green}{1}}} &
      {\textbf{\textcolor{green}{1}}}\\
      \hline
      {\textbf{\textcolor{green}{1}}} &
      {\textbf{\textcolor{green}{1}}} & {\textbf{\textcolor{red}{2}}} &
      {\textbf{\textcolor{red}{15}}} & {\textbf{\textcolor{green}{1}}} &
      {\textbf{\textcolor{green}{1}}}\\
      \hline
      {\textbf{\textcolor{green}{1}}} & {\textbf{\textcolor{green}{1}}} &
      {\textbf{\textcolor{green}{1}}} & {\textbf{\textcolor{green}{1}}} &
      {\textbf{\textcolor{green}{1}}} & {\textbf{\textcolor{green}{1}}}\\
      \hline
      {\textbf{\textcolor{green}{1}}} & {\textbf{\textcolor{green}{1}}} &
      {\textbf{\textcolor{green}{1}}} & {\textbf{\textcolor{green}{1}}} &
      {\textbf{\textcolor{green}{1}}} & {\textbf{\textcolor{green}{1}}}\\
    \end{tabular}
    The values equal to  {\textbf{\textcolor{green}{1}}} represent the extended domain.
  \end{center}
  \label{tab:pprop01}\caption{Image with an extended domain of size 2.}
\end{table}
An extended domain increases the performances of the image processing
algorithms.
Indeed, image processing algorithms often use the neighbors of the image
sites.
With an extended border it isn't necessary to test if the neighbors of an image
site are included in the image domain.
The size of the extended domain can differ (\verb+fixed+, \verb+infinite+ or
\verb+extendable+).



%% FIXME illustration


%%%%%%%%%%%%%%%%%%%%%%%%
%   ext_value          %
%%%%%%%%%%%%%%%%%%%%%%%%
\subsubsection{ext\_value}
\index{Property: ext\_value}

\begin{verbatim}
  ext_value:  any
               |
               +--- undefined
               |
               +--- single
               |
               +--- multiple
\end{verbatim}

If all the extended domain sites share the same value (in memory), this property
is set to ``single''.
If each site of the image extended domain has its proper value, this property
is set to ``multiple''.
Otherwise, this property is set to ``undefined''.


%%%%%%%%%%%%%%%%%%%%%%%%
%   ext_io             %
%%%%%%%%%%%%%%%%%%%%%%%%
\subsubsection{ext\_io}
\index{Property: ext\_io}

\begin{verbatim}

  ext_io:  any
            |
            +--- /some/
            |      |
            |      +--- read
            |      |
            |      +--- read_write
            +--- none
\end{verbatim}

This property describes  the read/write accessibility of the image extended
border values.
